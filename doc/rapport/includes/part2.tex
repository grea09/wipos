\lhead{User manual}
\section{User manual}

This part explains how to run our complete project, the three different parts.

\subsection{Running the air point program}

You have to compile the router program : in the router directory run \verb+make+
and then send the generated binary called \verb+router+ to the router via scp.

Log as root on the router and run the command 

\verb+# iwc monitor 1+

to enable the \verb+prism0+ interface.

Then, run our program using the following command :

\verb+# ./router+

This way, the air point listen to the RSSI messages and can reply to the server
requests.


\subsection{Positioning server}

% How to use the server
First we need to use Eclipse to import the project. Relocate the server.db SQLite file to the working directory in which the server will be launched (Often home directory).
Then go to Options > Servers and add a tomcat 7 server.
After that right click the project and affect the server to it. A simple click on run must lauch the server.


\subsection{Android}

% How to use the Android application
The Android application have two modes :
\begin{itemize}
    \item Measure mode
    \item Locate mode
\end{itemize}
You can swich between them by clicking the Mode button. In measure mode point your location then click accept.
In Location mode a pin appear where the server locate you.
