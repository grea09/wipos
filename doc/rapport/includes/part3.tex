\lhead{Analysis of the work}
\section{Analysis of the work}

\subsection{Positioning server}

% ____________________________________________________________________________
\subsection{Android terminal}

\subsubsection{Overview}

The application must communicate to the server in order to locate the android
terminal. 

\subsubsection{Classes}

The \verb+main+ class extends the \verb+Activity+ class. It set the layout, with
the buttons and the map, and interacts with the other classes to communicate
with the server.

The \verb+Locate+ class...

The \verb+Locate+ class...

The \verb+Common+ class...

\subsubsection{Layout}

The Android application shows a map (a picture), on which we display the
position of the terminal.

We have two modes, with corresponding buttons :

\begin{itemize}
    \item toggle button On : set visible \textit{Cancel} and \textit{Measure}
        buttons.
    \item toggle button Off : set visible \textit{Locate} button.
\end{itemize}

\begin{description}
    \item[Cancel] Actually, does nothing, but does it well.
    \item[Measure] Send a measure request via the \verb+Measure+ class.
    \item[Locate] Send a locate request to know the terminal position via the
        \verb+Locate+ class.
\end{description}

When the user touch the map, the cursor put itself where the user touched.

% ____________________________________________________________________________
\subsection{Wi-Fi access point}

\subsubsection{Short recall of the specifications}

The Wi-Fi access point should run a RSSI measurement software. This software
manages two tasks :

\begin{itemize}
    \item listening the devices and keeping their information in a list
    \item listening the message from the positioning server
\end{itemize}

This two tasks will be implemented as two threads, using \verb+pthread+.

\subsubsection{RSSI listener : Configuring the access point}

The first step of this part was to configure the AP (Access Point) to listen
the IRSSI messages. So we installed the pcap library on it and then we set the
monitoring mode thanks to :

\begin{verbatim}
root@openwrt$ wlc monitor 1
\end{verbatim}

This command add the \verb+prism0+ interface, which can be opened with the
pcaplib previously installed, to get the information from the frames.

\subsubsection{RSSI listener : Parsing the frame}

The software read the frame header. This give us, among other things,
the MAC address of the device emitting the message, and the RSSI value.

\subsubsection{RSSI listener : Listing the information}

So we receive a lot of RSSI, linked to some source addresses. In order to stock
this information, we implement a doubly linked list : a list of MAC address,
and each MAC address linked to a list of the corresponding RSSI values.

We recall the structure of the linked list : 

\begin{lstlisting} 
typedef struct RssiList {
	struct timeval rl_date ; // Expiration date
	int rl_rssi_value ;
	struct RssiList * rl_next ;
} RssiList ;

typedef struct DeviceList {
	unsigned char dl_mac_address [ 6 ] ;
	RssiList * dl_rssi_list ;
	struct DeviceList * dl_next ;
} DeviceList ;
\end{lstlisting}

We had to implement some functions in order to manage the list. In addition
from the functions from the subject, we had some of our own.

\begin{itemize}
    \item \verb+DeviceList* add_device (DeviceList ** l , unsigned char mac_addr [ 6 ]);+\\
        Add a device (represented by its MAC address to the linked list)
    \item \verb+void clear_device_list (DeviceList ** l);+\\
        Clear the list from its device, and so their linked RSSI 
    \item \verb+void add_rssi_sample (DeviceList * l, int rssi_value);+\\
        Add an RSSI value to the given device \verb+l+
    \item \verb+void clear_rssi_list (DeviceList * l);+\\
        Clear the RSSI values of the given device \verb+l+
    \item \verb+void delete_outdate (DeviceList ** l);+\\
        Delete all the outdated value from the linked list
    \item \verb+void delete_outdated (DeviceList * l, struct timeval current_time);+\\
        (Called by the previous function) Clear the outdated RSSI values from a
        given device
    \item \verb+DeviceList* is_known(DeviceList ** l , unsigned char mac_addr [6]);+\\
        Return null pointer if the device is unknown, or the pointer to already
        listed device if it is known. 
    \item \verb+void print (DeviceList** l);+\\
        Print in a human readable format the linked list
    \item \verb+double average (DeviceList ** l , unsigned char mac_addr [6]);+\\
        Return the average value of all the RSSI values of a given device.
\end{itemize}

The linked list wasn't a big deal in this project as we already had to
implement such structure in others UVs. We run it during a long time on the AP
and didn't get any problem so we assume that it just works correctly. 

\subsubsection{Communication with the server}

We receive requests from the server with this form :

\begin{lstlisting}
typedef struct {
  char op[5];
  double x, y;
  int mapID;
  int semi_colon_nb;
  unsigned char mac_addr[6];
  unsigned char my_mac_addr[6];
  double average;
} request_t ;
\end{lstlisting}

When we receive such request, we parse it in order to get the concerning MAC
address. Then we compute the average of the RSSI values linked to this address
so we can send the response to the server.








