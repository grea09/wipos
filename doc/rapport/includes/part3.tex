\lhead{Analysis of the work}
\section{Analysis of the work}

\subsection{Wi-Fi access point}

\subsubsection{Short recall of the specifications}

The Wi-Fi access point should run a RSSI measurement software. This software
manages two tasks :

\begin{itemize}
    \item listening the devices and keeping their information in a list
    \item listening the message from the positioning server
\end{itemize}

This two tasks will be implemented as two threads, using \verb+pthread+.

\subsubsection{RSSI listener : Configuring the access point}

The first step of this part was to configure the AP (Access Point) to listen
the RSSI messages. So we installed the pcap library on it and then we set the
monitoring mode thanks to :

\begin{verbatim}
root@openwrt$ wlc monitor 1
\end{verbatim}

This command add the \verb+prism0+ interface, which can be opened with the
pcaplib previously installed, to get the information from the frames.

\subsubsection{RSSI listener : Parsing the frame}

The software read the frame header. This give us, among other things,
the MAC address of the device emitting the message, and the RSSI value.

\subsubsection{RSSI listener : Listing the information}

So we receive a lot of RSSI, linked to some source addresses. In order to stock
this information, we implement a doubly linked list : a list of MAC address,
and each MAC address linked to a list of the corresponding RSSI values.

We recall the structure of the linked list : 

\begin{lstlisting} 
typedef struct RssiList {
	struct timeval rl_date ; // Expiration date
	int rl_rssi_value ;
	struct RssiList * rl_next ;
} RssiList ;

typedef struct DeviceList {
	unsigned char dl_mac_address [ 6 ] ;
	RssiList * dl_rssi_list ;
	struct DeviceList * dl_next ;
} DeviceList ;
\end{lstlisting}

We had to implement some functions in order to manage the list. In addition
from the functions from the subject, we had some of our own.

\begin{itemize}
    \item \verb+DeviceList* add_device (DeviceList ** l , unsigned char mac_addr [ 6 ]);+\\
        Add a device (represented by its MAC address to the linked list)
    \item \verb+void clear_device_list (DeviceList ** l);+\\
        Clear the list from its device, and so their linked RSSI 
    \item \verb+void add_rssi_sample (DeviceList * l, int rssi_value);+\\
        Add an RSSI value to the given device \verb+l+
    \item \verb+void clear_rssi_list (DeviceList * l);+\\
        Clear the RSSI values of the given device \verb+l+
    \item \verb+void delete_outdate (DeviceList ** l);+\\
        Delete all the outdated value from the linked list
    \item \verb+void delete_outdated (DeviceList * l, struct timeval current_time);+\\
        (Called by the previous function) Clear the outdated RSSI values from a
        given device
    \item \verb+DeviceList* is_known(DeviceList ** l , unsigned char mac_addr [6]);+\\
        Return null pointer if the device is unknown, or the pointer to already
        listed device if it is known. 
    \item \verb+void print (DeviceList** l);+\\
        Print in a human readable format the linked list
    \item \verb+double average (DeviceList ** l , unsigned char mac_addr [6]);+\\
        Return the average value of all the RSSI values of a given device.
\end{itemize}

The linked list wasn't a big deal in this project as we already had to
implement such structure in others UVs. We run it during a long time on the AP
and didn't get any problem so we assume that it just works correctly.

All the different parts were implemented.

\subsubsection{Communication with the server}

We receive requests from the server with this form :

\begin{lstlisting}
typedef struct {
  char op[5];
  double x, y;
  int mapID;
  int semi_colon_nb;
  unsigned char mac_addr[6];
  unsigned char my_mac_addr[6];
  double average;
} request_t ;
\end{lstlisting}

When we receive such request, we parse it in order to get the concerning MAC
address. Then we compute the average of the RSSI values linked to this address
so we can send the response to the server.

\subsection{Analysis}

As we said, this part of the project works correctly and all the different
parts were implemented. 

If we had a little more time and if we wanted to make something more usable, we
would make some changes. We would make it run in background (as a daemon for
example). Maybe we would add the possibility for the user to choose the port to
communicate, by specifying an argument to our executable.

\newpage

\subsection{Positioning server}


The application is divided in 3 packages. A \verb+Common+ class is used for configuration and generic fonctionalities.
\subsubsection{Logic}
The Database was implemented by using Sqlite. Sqlite is a light and effecient Database Management System, that can store its information in a sincle file. It don't need any configuration. The Server use an existing file, but can also generate the schema.
The logic package hold all the logical classes. Each class match a table in the database. They are configured to work with ORMLite in order to simplify database management. ORMLite is a simple java ORM (Object Relational Mapper) that can easily manage to seamlessly transform Table as if they where java objects. Some key classes have a \verb+fromRequest+ function. This function can initialise the object from splited request or response. These objects are very simple and have all atributes set as public as no control is needed.
Here is the Location class :
\begin{lstlisting}
@DatabaseTable
public class Location
{
	public Location()
	{
		super();
	}
	@DatabaseField(generatedId = true)
	public int id;
	@DatabaseField
	public double x;
	@DatabaseField
	public double y;
	@DatabaseField(foreign = true)
	public Map map;
	
	public String toString()
	{
		return x+";"+y+";";
	}

	public void fromRequest(String[] params)
	{
		if(params.length > 4)
		{
			x = Integer.decode(params[1]);
			y = Integer.decode(params[2]);
		}
	}
}
\end{lstlisting}

\subsubsection{Servlets}
As previously discuted, there are two servlets, \verb+Measure+ and \verb+Locate+. Each one start the \verb+UdpThread+.
\verb+Measure+ just send a request to all AP. \verb+Locate+ is a bit more complicated as it must wait for the AP responses and database filling. The \verb+Locate+ class has a static list of waiting servlets :
\begin{lstlisting}
private static HashMap<String, Locate> waitList = new HashMap<String, Locate>();
\end{lstlisting}
At start it verify if there are no pending request for same client mac address. after sending the request to AP, it insert itself in waiting list. Then the servlet invoke \verb+wait(timeout)+ in order to wait for all the APs responses. The \verb+UdpThread+ notify for each response the servlert corresponding to the client mac address.
The \verb+Locate+ have also a locate function that must retrieve the nearest measurment from the database.
Here is the single result request that retrieve it :
\begin{lstlisting}
select id
from TempRssi as T
	inner join Rssi as R
	on T.accessPoint = R.accessPoint
where T.clientMac = '?'
group by 1
order by (avg(abs(R.average - T.average))) asc limit 1
\end{lstlisting}
The ? parameter is the client mac address. This is the only request of the project as there are no need to use those with an ORM.

\subsubsection{Thread}
The \verb+UdpThread+ class is used to gather all APs response and to fill the database with gathered data. Eventualy, if the response does not contains the additional data (location response) the \verb+UdpThread+ notify the waiting servlet.

\newpage
\subsection{Android terminal}

As we already worked on Android development in other UVs or for our personal
interest, we were already familiarized with this environment.

\subsubsection{Overview}

The application must communicate to the server in order to locate the android
terminal.

\subsubsection{Classes}

The \verb+main+ class extends the \verb+Activity+ class. It set the layout, with
the buttons and the map, and interacts with the other classes to communicate
with the server.

The \verb+Measure+ class extends AsyncTask in order to avoid calling network functionality in main thread. It send a burst of broadcast packets then send our location to the server. 

The \verb+Locate+ class is the same exepts that it ask for position and then return it.

The \verb+Common+ class hold constants like server URL and port to configure the application.

\subsubsection{Layout}

The Android application shows a map (a picture), on which we display the
position of the terminal.

We have two modes, with corresponding buttons :

\begin{itemize}
    \item toggle button On : set visible \textit{Cancel} and \textit{Measure}
        buttons.
    \item toggle button Off : set visible \textit{Locate} button.
\end{itemize}

\begin{description}
    \item[Cancel] Actually, does nothing, but does it well.
    \item[Measure] Send a measure request via the \verb+Measure+ class.
    \item[Locate] Send a locate request to know the terminal position via the
        \verb+Locate+ class.
\end{description}

When the user touch the map, the cursor put itself where the user touched.
Then the user can accept or cancel measure.
Once accepted the measure is send as parameter of the Measure class to register it.
The locate button trigger a location request by using the Locate class. The location is returned to caller and the cursor is placed in the location returned by the server.








